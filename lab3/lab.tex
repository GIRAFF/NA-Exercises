\documentclass[oneside, final, 12pt]{extarticle}
\usepackage[utf8]{inputenc}
\usepackage[russian]{babel}
\usepackage{vmargin}
\usepackage{listings}
\usepackage{graphicx}
\usepackage{amsmath}
\usepackage{amssymb}
\usepackage{longtable}
\setpapersize{A4}
\setmarginsrb{2cm}{2cm}{2cm}{2cm}{0pt}{0mm}{0pt}{13mm}
\usepackage{indentfirst}%красная строка
\sloppy

\begin{document}
\begin{titlepage}
	\begin{centering}
		\textsc{Министерство образования и науки Российской Федерации}\\
		\textsc{Новосибирский государственный технический университет}\\
		\textsc{Кафедра теоритической и прикладной информатики}\\
	\end{centering}
	%\centerline{\hfill\hrulefill\hrulefill\hfill}
	\vfill
	\vfill
	\vfill
	\Large
	\centerline{Лабораторная работа №3}
	\centerline{по дисциплине "<Численные методы">}
	\centerline{\bfТрёхшаговые итерационные методы решения СЛАУ}
	\centerline{\bfс предобуславливанием}
	\normalsize
	\vfill
	\vfill
	\vfill
	\begin{flushleft}
		\begin{minipage}{0.3\textwidth}
			\begin{tabular}{l l}
				Факультет: & ПМИ\\
				Группа: & ПМИ-41\\
				Студент: & Кислицын И. О.\\
				Преподаватели: & Персова М. Г.\\
				~ & Задорожный А. Г. 
			\end{tabular}
		\end{minipage}
	\end{flushleft}
	\vfill
	\vfill
	\begin{centering}
		Новосибирск\\
		2016\\
	\end{centering}
\end{titlepage}
\setcounter{page}{2}
\lstset{
	breaklines=\true,
	%frame=single,
	basicstyle=\footnotesize\ttfamily,
	tabsize=2,
	showspaces=\false,
	breaklines=\true,
	breakatwhitespace=\true,
	%escapeinside={[}{]},
	%inputencoding=utf8x,
	extendedchars=\true,
	keepspaces=\true,
	language=Haskell
}
\section{Цель работы}

Изучить особенности реализации трехшаговых итерационных методов для СЛАУ с разреженными матрицами. Исследовать влияние предобусловливания на сходимость изучаемых методов на нескольких матрицах большой (не менее 10000) размерности.

\section{Программа}

\lstset{caption=main.cc}
\lstinputlisting[language=C++]{main.cc}

\section{Исследования}

\subsection{Матрица из лаб. 2}

\begin{tabular}{|c|c|c|} \hline
	\bf\(x^*-x\) & \bfИтер. & \bfНевязка \\ \hline
	\(\begin{aligned}
		& -5.609291e-12 \\
		& 2.3752111e-12 \\
		& -3.916867e-13 \\
		& 4.5741189e-14 \\
		& -3.730349e-14 \\
		& 9.7699626e-15 \\
		& -3.286260e-14 \\
		& 3.4638958e-12 \\
		& -4.440892e-13 \\
		& 1.5276668e-13 \\
	\end{aligned}\) & \(9\) & \(1.52199e-12\) \\ \hline
\end{tabular}

\subsection{Матрицы больших размерностей}

\begin{tabular}{|c|c|c|} \hline
	\bfРазм. & \bfИтер. & \bfНевязка \\ \hline
	\(945\) & \(390\) & \(9.21063e-21\) \\ \hline
	\(4545\) & \(2005\) & \(9.23667e-21\) \\ \hline
\end{tabular}

\end{document}
